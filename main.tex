%Das ist die Hauptdatei Hier startet der Compiler.
%Das Dokument besteht immer aus Beginn und end. Davor ist Präampel und enthält Konfigurationen

\documentclass[a4paper, 12pt]{article}

\usepackage[ngerman]{babel} %Sprachpaket Deutsch
\usepackage[T1]{fontenc} %Anpassung

\title{LaTeX}
\author{Ich}
\date{\today}

%neue Befehle:
\newcommand{\Befehlsname}[1][Standardwert]{<Definition>}
%Standardanzahl der Parameter ist 0, nicht mehr als 9!
%Standardwert/Standardwert, falls er nicht mitgegeben wird, ist optional
%Definition, was halt passieren soll
%in der Definition werden die Parameter mit #1, #2, #3, ... usw. angesprochen
%überschreiben mit renewcommand

%Beispiele
\newcommand{\hallo}{Hallo Welt!}
%Ouput: "Hallo Welt!"

\newcommand{\gruss}[1]{Hallo, #1!}

\newcommand{\email}[2][E-Mail-Adresse]{Kontakt: #1, Name: #2}
%mit optionalen Parameter

\begin{document}

\gruss{Peter} %Output: "Hallo, Peter!"
\email{Hans} %Output: "Email-Adresse Hans"; ich könnte auch ersten Paramter belegen

\maketitle
\tableofcontents %Inhaltsverzeichnis, macht der automatisch

\section{Dies ist ein Kapitel} %Kapitelabschnitt

Ein bisschen Text.

\section{Dies ist ein weiteres Kapitel}

Ein bisschen Text findet hier statt. %"Quelltext" aus anderer Datei an genau dieser Stelle eingefügt auf neuer Seite
\include{Dok/Dok2} %Aus einem Unterordner
\section{Dies ist ein weiteres Kapitel}

Ein bisschen Text findet hier statt. %ohne neuer Seite
%besonders sinnvoll für configs

\end{document}