%Das ist die Hauptdatei. Hier startet der Compiler.
%Das Dokument besteht immer aus Beginn und end. Davor ist Präampel und enthält Konfigurationen

\documentclass[a4paper, 12pt]{article}

\usepackage[ngerman]{babel} %Sprachpaket Deutsch
\usepackage[T1]{fontenc} %Anpassung
\usepackage{pdfpages} %für PDFs einfügen
\usepackage[backend=biber, style=authoryear]{biblatex} %für die bib-Datei

\title{LaTeX}
\author{Ich}
\date{\today}


\addbibresource{deineLiteratur.bib} %Einbinden der bib-Datei aus Zotero


%globale Konstanten erstellen:
\newcommand{\setmyyear}[1]{\newcommand{\myyear}{#1}}
\setmyyear{2025}
\myyear %Aufruf

%wenn mehrmals nutzbar sein soll
\newcommand{\myyear}{} % initiale Definition
\newcommand{\setmyyear}[1]{\renewcommand{\myyear}{#1}}




%neue Befehle:
\newcommand{\Befehlsname}[1][Standardwert]{<Definition>}
%Standardanzahl der Parameter ist 0, nicht mehr als 9!
%Standardwert/Standardwert, falls er nicht mitgegeben wird, ist optional
%Definition, was halt passieren soll
%in der Definition werden die Parameter mit #1, #2, #3, ... usw. angesprochen
%überschreiben mit renewcommand

%Beispiele
\newcommand{\hallo}{Hallo Welt!}
%Ouput: "Hallo Welt!"

\newcommand{\gruss}[1]{Hallo, #1!}

\newcommand{\email}[2][E-Mail-Adresse]{Kontakt: #1, Name: #2}
%mit optionalen Parameter


%flexiblere Befehle:
\usepackage{xparse}
\NewDocumentCommand{\befehl}{O{} m}{\ifblank{#1}{#1}{#2}}
%O{Startwert} ... optional
%m... Pflicht
%in def Zugriff auf Args mit #1 etc.
%letzte Klammer ist def, dort kann man "alles schreiben", auch Text
%jeweiligen Parameter prüfen: \ifblank : 1:Welcher Parameter 2:Wenn wahr 3:Wenn falsch
%bei Aufruf \befehl[]{} []...optional (kann auch weg, wenn kein Wert dahinter (1 kann weg, 2 nicht) [1]{Wert}[2][66]) {}...Pflicht



\begin{document}

\gruss{Peter} %Output: "Hallo, Peter!"
\email{Hans} %Output: "Email-Adresse Hans"; ich könnte auch ersten Paramter belegen

\maketitle
\tableofcontents %Inhaltsverzeichnis, macht der automatisch

\section{Dies ist ein Kapitel} %Kapitelabschnitt
\subsection{Titel} %für Untersüberschrift (\subsubsection{Titel} usw.)
\paragraph{Titel} %für Teilüberschriften (auch hier sub, subsub, etc.)
%mit Stern werden sie nicht im Inhaltsverzeichnis aufgeführt \section*{}

Ein bisschen Text.

Für direkte Zitate, kann man die eingebaute Funktion Footcite \footcite[2]{psgithub} benutzen.
Unter der LaTeX-Vorlage der DHSN sind ein paar komplexere Funktionen zu finden.

\section{Dies ist ein weiteres Kapitel}

Ein bisschen Text findet hier statt. %"Quelltext" aus anderer Datei an genau dieser Stelle eingefügt auf neuer Seite
\include{Dok/Dok2} %Aus einem Unterordner
\section{Dies ist ein weiteres Kapitel}

Ein bisschen Text findet hier statt. %ohne neuer Seite
%besonders sinnvoll für configs

\includepdf[pages=-]{anhang.pdf}
%pages=- bedeutet alle Seiten einfügen
%Bestimmte Seiten: pages=1-3 --> Seiten 1 bis 3
%pages={1,3,5} --> Nur diese Seiten
%Optional: pagecommand={} entfernt die Seitenzahlen
%Pfad beachten!

\end{document}

\printbibliography %damit das Literaturverzeichnis geprintet wird.